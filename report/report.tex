\documentclass[12pt,leqno,a4paper]{article}
% Minimalbeispiel f\"ur Diplom- oder Studienarbeit in Latex
% OHNE GEWAEHR
% Antje Schweitzer, Juli 2011 & Juli 2014
% folgende Dateien gehoeren zu diesem Beispiel:
% - thesis.tex (der eigentlich Text)
% - references.bib (die "Datenbank" mit den Literaturverweisen im Bibtex-Format)
% - example.pdf (eine Beispielgrafik)
% damit kann man dann nach der Anleitung unten folgendes Dokument erzeugen:
% - thesis.pdf 

% Minimal example for student papers in LaTeX, can be used as a template
% without warranty
% (English "translation" by Nils Reiter, July 2014)
% You need multiple files for this example to work:
% - thesis.tex (this file)
% - references.bib (a BibTeX file containing the bibliographic entries)
% - example.pdf (an example figure)

% ANLEITUNG -- INSTRUCTIONS
% To create a PDF, please run the following steps:
% - pdflatex thesis.tex
% - bibtex thesis
% - pdflatex thesis.tex
% - pdflatex thesis.tex
% (yes, some steps need to be run multiple times, for reasons)


\usepackage{natbib}
\usepackage{epsfig}
\usepackage{booktabs}
\usepackage{url}
\usepackage{xeCJK}
\usepackage[toc,page]{appendix}
\setCJKmainfont{SimSun}

% recommended: uncomment following line if you write in German 
%\usepackeage{german}  


\renewcommand{\baselinestretch}{1.3}
\parskip = \medskipamount
\frenchspacing
\bibpunct[; ]{(}{)}{;}{a}{,}{;}


\newcommand{\Titel}{Lexical and Syntactical Analysis \\of Chinese Idioms}

\includeonly{sect1, sect2, sect3, sect4, sect5, appendix}


\begin{document}

\begin{titlepage}
  \large
  \begin{center}
    Institut f\"ur Maschinelle Sprachverarbeitung\\
    Universit\"at Stuttgart\\
    Pfaffenwaldring 5B\\
    D-70569 Stuttgart\\    
      \vfill
    {\LARGE \bf \Titel} \\
    \vspace{2cm}
    Xiang Yu, Weimeng Zhu\\
    Term Paper for: \\
    Detecting and Classifying Multi-Word Expressions \\
      \vfill
    \begin{tabular}[t]{lr}
    % {Pr\"ufer:} & Prof. Dr. Frida Maier\\
    %  & Prof. Dr. Max M\"uller\\
    % {Betreuer:} & Dr. Harald Schmidt\\ 
    \\
    \\
    % {Beginn der Arbeit:} & 01.03.2010\\
    {Due Date:} & 15.03.2015\\
    \end{tabular}
  \end{center}

  \normalsize
\end{titlepage}

\newpage
\thispagestyle{empty}


% \noindent\textbf{Erkl\"arung (Statement of Authorship)}\\


% \noindent Hiermit erkl\"are ich, dass ich die vorliegende Arbeit selbst\"andig verfasst habe und dabei keine andere als die angegebene Literatur verwendet habe. Alle Zitate und sinngem\"a\ss{}en Entlehnungen sind als solche unter genauer Angabe der Quelle gekennzeichnet

% \noindent (This text is the result of my own work. Material from the published or unpublished work of others, which is referred to in the text, is credited to the author in the text)\\[2cm]


% \noindent (Marlene Mustermann)


% \newpage

\tableofcontents
\newpage

\section{Introduction}
Chinese idioms (成语), are multi-word expressions in Chinese language, mostly derived from history or classical literature. There are many resources for Chinese idioms, with the collection size ranges from 5000 to 20000, depending on the definition.

% Most of the idioms consist of four characters/words, carrying more meaning than the sum of the words, since they usually refer to a story, one can not understand the meaning of the idiom without knowing the underlying story.

The origin of the idioms ranges starts from more than 2000 years ago, therefore the use of syntax and meaning of vocabulary can be quite different from modern standard Chinese. 

In this report, we focus our research on two aspect on Chinese idioms.
One is the frequent nouns in Chinese idioms. In concrete, we first annotated the part-of-speeches of the words used in the idioms, then sorted the nouns by their frequencies, and compared them to the most frequent nouns in German and Spanish \citep{mahlow:2013, Duden:2008, Seco:2004}. 
The other aspect is the frequent syntax patterns of Chinese idioms, since we have annotated the part-of-speeches, we could explore the structures and how they differ from modern Chinese syntax.

The results in the first part confirmed that most of the frequent nouns are shared in the MWEs of the three languages, we then tried to explain the presence and absence of some words from different aspects.

In the second part we found the most remarkable feature in the syntactic structure, namely the parallelism. 

The report is structured as follows: Section~\ref{anno} describes the annotation guidelines and procedures, Section~\ref{nouns} shows the statistics of the top nouns in Chinese idioms and the comparison to German and Spanish MWEs, Section~\ref{discuss} discusses the possible reasons for the difference between them, Section~\ref{pattern} analyzes the syntactic structures of the idioms. A glossary of the frequent words are given in the appendix.

\section{Annotation}\label{anno}
The idioms we used are from a dictionary \citep{song:2014}, containing 13000 idioms. 
Since the words in the idioms are unannotated, we first have to annotate them. We designed the {\em NAIV} annotation scheme, which  groups the words into four main classes: noun (N), adjective/adverb (A), numeral (I) and verb (V), and those not belong to either of the classes are annotated as functional (F). 

The annotation is not based on word token but word type, i.e. all instances of a word are annotated as the same class. No doubt this will hurt the accuracy, but annotating each word token would require hundreds of hours, therefore we could only first take the approximation and leave the more fine-grained annotation to future works.

The annotation procedure works as follows: (1) sort the words by their frequency, (2) for the words with frequency greater than 9 (about 1200 in total), randomly sample 10 idioms where the word appear, and assign a class for it, (3) each person do the whole annotation once and then compare the result together, discuss and repeat step (2) for conflict results.


% Introduction


\section{Annotation}\label{anno}
The idioms we used are from a dictionary \citep{song:2014}, containing 13000 idioms. 
Since the words in the idioms are unannotated, we first have to annotate them. We designed the {\em NAIV} annotation scheme, which  groups the words into four main classes: noun (N), adjective/adverb (A), numeral (I) and verb (V), and those not belong to either of the classes are annotated as functional (F). 

The annotation is not based on word token but word type, i.e. all instances of a word are annotated as the same class. No doubt this will hurt the accuracy, but annotating each word token would require hundreds of hours, therefore we could only first take the approximation and leave the more fine-grained annotation to future works.

The annotation procedure works as follows: (1) sort the words by their frequency, (2) for the words with frequency greater than 9 (about 1200 in total), randomly sample 10 idioms where the word appear, and assign a class for it, (3) each person do the whole annotation once and then compare the result together, discuss and repeat step (2) for conflict results.


A major problem we meet during the annotation is the frequent phenomenon of functional shift in ancient Chinese, which changes the syntactic function of a word. For example, the word 衣, which originally means the noun {\em cloth}, also used exclusively as noun in modern Chinese, is frequently shifted as verb {\em to dress} in ancient Chinese and in idioms. Therefore, we have to observe the function of this word in the 10 sample idioms and take the majority as its class. Mistakes could arise from this sampling method, but it is the only feasible way to annotate this amount of words in limited time.



\section{Experiments and Results}\label{nouns}

\subsection{Results for Top Nouns in Chinese and German}\label{firstresult}
After annotation, we analyzed all idioms in the dictionary and found the most frequent nouns occurred. Then we aligned the top 50 Chinese nouns and top 50 German nouns to inspect the differences in distribution. 

We found top 50 German nouns that can be aligned to the top 50 Chinese nouns and shown in Table~\ref{tabletncg}. The table shows the order for each word in the top list (e.g. Mann is the 24th in German list). From the table, we can find 29 out of 50 nouns are aligned. The number is reasonable since we have only 50 vs. 50 alignment. In the table, it is shown that the nature phenomena are more frequent while they are less frequent in German. 


\begin{table}[h!]
\centerline{
\begin{tabular}{|c|c|c|c|c|c|c|c|c|c|}
\hline
\hline
1 & 2 & 3 & 4 & 5 & 6 & 7 & 8 & 9 & 10 \\
人 & 心 & 天 & 风 & 山 & 头 & 地 & 水 & 日 & 马 \\
\hline
24 & 5 & 23 & 25 &  & 3 & 26 & 16 & 7 & 50 \\
Mann & Herz & Himmel & Wind &  & Kopf & Boden & Wasser & Tag & Pferd \\
\hline
\hline
11 & 12 & 13 & 14 & 15 & 16 & 17 & 18 & 19 & 20 \\
金 & 目 & 事 & 口 & 道 & 手 & 云 & 月 & 气 & 门 \\
\hline
36 & 2 &  & 21 & 10 & 1 &  &  & 27 & 42 \\
Geld & Auge &  & Mund/Maul & Weg & Hand &  &  & Luft & T\"{u}r \\
\hline
\hline
21 & 22 & 22 & 23 & 24 & 24 & 24 & 25 & 26 & 26 \\
足 & 虎 & 面 & 火 & 身 & 流 & 子 & 玉 & 声 & 情 \\
\hline
6 &  & 31 & 34 & 29 &  & 33 &  &  & \\
Fu\ss &  & Gesicht & Feuer & Leib &  & Kind &  &  & \\
\hline
\hline
27 & 28 & 28 & 28 & 29 & 30 & 31 & 32 & 32 & 33 \\
食 & 花 & 发 & 名 & 海 & 雨 & 世 & 家 & 年 & 龙 \\
\hline
  &  & 45 &  &  &  & 18 & 28 &  & \\
  &  & Haar &  &  &  & Welt & Haus &  & \\
\hline
\hline
34 & 35 & 36 & 37 & 38 & 39 & 40 & 41 & 41 & 42 \\
力 & 意 & 色 & 衣 & 神 & 眼 & 耳 & 国 & 文 & 公 \\
\hline
  & 40 &  &  & 35 & 2 & 4 &  & 11 & \\
  & Sinn &  &  & Seele & Auge & Ohr &  & Wort & \\
\hline
\hline
\end{tabular}}
\caption{\label{tabletncg} Top 50 Nouns Alignment for Chinese and German (ordered in Chinese)}
\end{table}


Table~\ref{tabletngc} aligns Chinese nouns to top 50 German nouns. Since we use a longer list for Chinese (containing top 500 Chinese nouns), we have 41 out of 50 matched. Due to this huge list and translation, we have more than one Chinese noun matched to a German noun, but we also provide the order for each words. It is shown that the body parts comes more than other nouns in German, while most of them can also be found in Chinese with relatively high frequency. 

\begin{table}[h!]
\centerline{
\begin{tabular}{|c|c|c|c|c|c|c|c|c|c|}
\hline
\hline
1 & 2 & 3 & 4 & 5 & 6 & 7 & 8 & 9 & 10 \\
Hand & Auge & Kopf & Ohr & Herz & Fu\ss & Tag & Finger & Gott & Weg \\
\hline
16 & 12/46 & 5 & 47 & 2 & 21/132 & 9 &  & 437 & 15/101 \\
手 & 目/眼 & 头 & 耳 & 心 & 足/脚 & 日 &  & 圣 & 道/路 \\
\hline
\hline
11 & 12 & 13 & 14 & 15 & 16 & 17 & 18 & 19 & 20 \\
Wort & Bein & Arsch & Nase & Hals & Wasser & Zeit & Welt & Teufel & Leben \\
\hline
48/77/166/229 &  &  & 464 &  & 8 & 52 & 37 & 118 & 82 \\
文/语/字/词 &  &  & 鼻 &  & 水 & 时 & 世 & 鬼 & 命 \\
\hline
\hline
20 & 21 & 22 & 23 & 24 & 25 & 25 & 26 & 27 & 28 \\
Hund & Mund & Licht & Himmel & Mann & Tod & Wind & Boden & Luft & Haus \\
\hline
105/151 & 14/176 & 67 & 3 & 1 &  & 4 & 7 & 19 & 38/229 \\
狗/犬 & 口/嘴 & 光 & 天 & 人 &  & 风 & 地 & 气 & 家/屋 \\
\hline
\hline
29 & 29 & 30 & 31 & 31 & 31 & 32 & 33 & 33 & 34 \\
Leib & T\"{u}r & Ende & Tisch & Blut & Gesicht & Seite & R\"{u}cken & Kind & Sack \\
\hline
25 & 20 & 166 & 360 & 128 & 22 & 312 & 138 & 25/218 & \\
身 & 门 & 终 & 案 & 血 & 面 & 边 & 背 & 子/儿 & \\
\hline
\hline
34 & 35 & 36 & 36 & 37 & 37 & 38 & 38 & 39 & 40 \\
Feuer & Seele & Geld & Zunge & Brot & Maul & Tasche & Dreck & St\"{u}ck & Sinn \\
\hline
24 & 45 & 11/332 & 89 &  & 14/176 &  & 464 &  & 42 \\
火 & 神 & 金/钱 & 舌 &  & 口/嘴 &  & 污 &  & 意 \\
\hline
\hline
\end{tabular}}
\caption{\label{tabletngc} Top 50 Nouns Alignment for Chinese and German (ordered in German)}
\end{table}


\subsection{Results for more language}
We also compared the list of nouns in Chinese idioms with those in Spanish MWEs, the results are shown in Figure~\ref{fig1} to Figure~\ref{fig5}. 

\begin{figure}[h]
\centerline{\epsfig{figure=figure1,width=120mm}}
\caption{{\it Top 20 Frequent Noun Distribution in Chinese and German (ordered in Chinese)}}
\label{fig1}
\end{figure}

Figure~\ref{fig1} shows the frequency for the top 20 Chinese nouns aligned with German nouns (up to 25th Chinese noun). The y-axis shows translation-Chinese-German for each item. As is discussed in Section~\ref{firstresult}, body parts are more frequent in German compared to nature phenomena in Chinese. 

\begin{figure}[h]
\centerline{\epsfig{figure=figure2,width=120mm}}
\caption{{\it Top 20 Frequent Noun Distribution in Chinese and Spanish (ordered in Chinese)}}
\label{fig2}
\end{figure}

Figure~\ref{fig2} shows the figure for Chinese and Spanish. The alignment is harder than with German since we used up to 47th Chinese noun to get 20 matches. Compared to German version, in spite of less match in most frequent nouns, the distribution for the matches are more similar in respect of the absolute frequency number. 

\begin{figure}[h]
\centerline{\epsfig{figure=figure3,width=120mm}}
\caption{{\it Top 20 Frequent Noun Distribution in Chinese and German (ordered in German)}}
\label{fig3}
\end{figure}

\begin{figure}[h]
\centerline{\epsfig{figure=figure4,width=120mm}}
\caption{{\it Top 20 Frequent Noun Distribution in Chinese and Spanish (ordered in Spanish)}}
\label{fig4}
\end{figure}

Figure~\ref{fig3} and Figure~\ref{fig4} are the alignment for German with Chinese and Spanish with Chinese. The result shows no significant difference. But this time, German is matched up to 39th while Spanish was up to 48th. Compared to Figure~\ref{fig1}, the alignment is harder here. One interesting thing is that the frequency in Chinese are much higher than the other two languages for the same noun. 

\begin{figure}[t]
\centerline{\epsfig{figure=figure5,width=120mm}}
\caption{{\it Overlapped top nouns between languages}}
\label{fig5}
\end{figure}

In Figure~\ref{fig5}, we analyzed the number of overlapped nouns between languages with the list. It is obvious that European languages have more matched nouns than across continents. And it is also interesting that the matched nouns between Chinese and German are always more than between Chinese and Spanish, although non of them are larger than between German and Spanish. It is reasonable that the culture and lifestyle are completely different across Europe and East Asia which made such difference, but the more nouns we take into consideration, the less difference we have between languages. In a word, although culture and lifestyle do matter, but the use of nouns are somehow having a shared list regardless of those reasons. 


\section{Discussion}\label{discuss}

In this section, we try to find the reasons for the differences in the distribution of frequent nouns in Chinese, German and Spanish. 

\subsection{Stylistic Reason}
It is significant that body parts are more frequent in German and Spanish but nature phenomena in Chinese. One reason may be that the MWEs in German and Spanish are taken from colloquial situations, where the body nouns are best to describe and make metaphor for daily life. In contrast, Chinese idioms derives mostly from literatures, where the nature are more preferred. 

More on this kind of preferences, one may find many nouns like 花(flower), 雨(rain), 云(cloud), 月(moon), 山(mountain) as frequent Chinese nouns. In ancient Chinese literature, such image words are usually used for making metaphor for emotions. But these words are not as the same usage in German and Spanish. 

\subsection{Religion Reason}
In German and Spanish, we all found word for god (German: Gott, Spanish: dios), but we didn't find any related noun in Chinese. The simple reason for this is the religion. God is a noun for Christian, while Ancient Chinese had religion of Buddhism. The noun 佛(Buddha) is also in the Chinese list but with a relatively low frequency (the 223th, with 25 occurrences). This low frequency is also reasonable in our Chinese idiom dictionary because the idioms came from 200B.C. to 20th century, but Buddhism widely spread in Ancient China after 500A.D., and it never became the official national religion in any dynasty of China. Another traditional religion in Ancient China was Taoism, the related noun for it was 道. Unfortunately, this noun is polysemous with other meanings as road, method or doctrine. Although 道 is very frequent in the corpus (15th noun with 127 occurrences), we cannot assign it as a religion related noun. 

\subsection{Natural Reason}
We also found many nouns for animals in all the three languages. But the kind of frequent nouns are different in distributions among the languages. We found dog in all languages (Chinese: 狗/犬, German:Hund, Spanish: perro), but we had tiger(虎) only in Chinese, and cat(gato) only in Spanish, and also the fictional creature, dragon(龙), only in Chinese. 
For the animal dog, it appears in all countries of the three, and also very common in the daily life. So idioms tend to contain them in the sayings. But for the animal tiger, which lives only in Asian, surely does not appear in WMEs in German and Spanish. 
% For the animal cat, it is also a common animal in the three countries, but only a frequent word in Spanish. One reason may be that cat is one of the most favorite animal in Spain. But for the culture reason, cat is usually used as evil in Chinese. So you won't find many Chinese idioms containing cats, but a lot in Spanish. 

We also found an interesting noun distribution for food. We have bread (German: Brot, Spanish: pan) in German and Spanish, but no matching noun in Chinese. It is known that bread is not a traditional food in China, but it is in Europe. So it is obvious that this noun is so frequent in German (45th with 37 occurrence) and Spanish (25th with 58 occurrence), but no occurrence in Chinese. Instead of bread, we have 饭(rice/food, 193th) and 米(rice, 299th) in the Chinese list, but with rather low frequency, again, because in literatures, food are less concerned than in colloquial expressions. 


\section{Pattern Analysis}\label{pattern}
After analyzing the frequent words in different languages, we also tried to find frequent pattern in Chinese idioms. One important feature of Chinese idioms is that most of the idioms are constructed with four characters (a.k.a. word in idiom), so it is easy to find out pattern for constructing idioms and it would also be important to check the patterns to find out how those four-character idioms are constructed. 

\subsection{Pattern Extraction and Results}
After annotation, we got word type for frequent words (frequency larger than 9). So we look only for those idioms containing only four frequent words in the dictionary. The results are shown in Table~\ref{tablepaex} (translation for examples are not provided). In the `Pattern' column of the table, n for noun, v for verb, a for adverb and adjective, and i for numeral. 

\begin{table}[htb]
\caption{\label{tablepaex} {\it Top 20 Frequent Patterns and Example Idioms (ordered in frequency)}}
\centerline{
\begin{tabular}{|c|c|c|c|}
\hline
Order & Pattern & Frequency & Examples \\
\hline
1 & vnvn & 480 & 争权夺利	, 借刀杀人 \\
\hline
2 & nnnn & 236 & 人山人海, 唇枪舌剑 \\
\hline
3 & anan & 189 & 大头小尾	, 白山黑水 \\
\hline
4 & nvnv & 175 & 云消雾散	, 土崩瓦解 \\
\hline
5 & nana & 143 & 人多嘴杂	, 地久天长 \\
\hline
6 & nnvn & 121 & 井井有条, 日月入怀 \\
\hline
7 & anvn & 118 & 先声夺人	, 小鸟依人 \\
\hline
8 & annn & 94 & 满城风雨, 满面春风 \\
\hline
9 & vnvv & 87 & 临阵脱逃, 倾巢出动 \\
\hline
10 & vnnn & 75 & 看人眉眼	, 过眼烟云 \\
\hline
11 & anav & 70 & 上情下达	, 冷眼静看 \\
\hline
11 & inin & 70 & 一丝一毫	, 千山万水 \\
\hline
13 & nnan & 69 & 道边苦李	, 鼎鼎大名 \\
\hline
14 & vvvv & 65 & 生死存亡	, 起承转合 \\
\hline
14 & vava & 65 & 居高临下	, 弃暗投明 \\
\hline
16 & nvnn & 62 & 天随人愿	, 声振林木 \\
\hline
17 & vnan & 61 & 垂名青史	, 识途老马 \\
\hline
18 & avav & 56 & 南征北战, 寡闻少见 \\
\hline
19 & nnav & 55 & 人神共愤	, 原形毕露 \\
\hline
20 & nnvv & 49 & 利害得失	, 儿女成行 \\
\hline
20 & vvvn & 49 & 招摇过市, 凿凿有据 \\
\hline
\end{tabular}}
\end{table}

\subsection{Discussion for Patterns}
Table~\ref{tablepaex} shows top 20 frequent pattern in the dictionary. it is obvious that most of the patterns are somehow belong to parallelism. Half of them (10 patterns) are in `AABB' or `ABAB', and 3 of them are `ABCB' or `ABAC', only one third of them are in unbalanced structure. 

Since that most of the idioms are from Ancient Chinese, it is reasonable that they are quite in the preference to parallelism. Parallelism is one of the most used expression technique for Ancient Chinese literature. Another reason might be that it would be easy to be like in parallelism for the four-word structure. And parallelism is easier for people who use the idioms and read the idioms to remember and recognize them. 

When finding example for patterns and checking the results, we found that it was hard to find an example for some frequent pattern, for example, `annn' and `vvvn'. One reason for this is that most idioms are in parallelism, for those unbalanced patterns, it shouldn't be frequent in the idioms. Another reason is the limitation of NAIV annotation. We only annotated frequent words and didn't provide different annotations for polysemy. So there are many idioms belong to wrong pattern because of the annotation mistake. 

We also tried to use wild cards to extract more kind of patterns, but due to the four-word structural limitation, the results are not satisfying. In the future, we will produce more analysis after getting a more rigorous annotated corpus for Chinese idioms. 


%%%%%%%%%%%%%
% Bibliographie
\bibliographystyle{plainnat}
\bibliography{references}

\appendix
\setcounter{secnumdepth}{0}
\section{Appendix}
% In Table~\ref{tabletrans1}, Table~\ref{tabletrans2}, and Table~\ref{tabletrans3}, we provide the translations for Chinese, German and Spanish words we use for our alignment and analysis. All words are translated into English and align across languages based on the translation. 

\begin{table}[htb]
\centerline{
\begin{tabular}{|c|c|c|c|c|}
\hline
人 & 心 & 天 & 风 & 山 \\
human & heart & sky/day & wind & mountain \\
\hline
头 & 地 & 水 & 日 & 马 \\
head & earth & water & day/sun & horse \\
\hline
金 & 目 & 事 & 口 & 道 \\
money/gold & eye & thing & mouth & road/method/doctrine \\
\hline
手 & 云 & 月 & 气 & 门 \\
hand & cloud & moon & air & door \\
\hline
足 & 虎 & 面 & 火 & 身 \\
foot & tiger & face & fire & body \\
\hline
流 & 子 & 玉 & 声 & 情 \\
current & son/mister & jade & sound & feeling \\
\hline
食 & 花 & 发 & 名 & 海 \\
food & flower & hair & name & sea \\
\hline
雨 & 世 & 家 & 年 & 龙 \\
rain & world & home & year & dragon \\
\hline
力 & 意 & 色 & 衣 & 神 \\
strength & meaning & color & clothes & spirit \\
\hline
眼 & 耳 & 国 & 文 & 公 \\
eye & ear & country & article/word & public/duke \\
\hline
物 & 脚 & 路 & 语 & 字 \\
thing & foot & road & word & word \\
\hline
词 & 狗 & 犬 & 嘴 & 屋 \\
word & dog & dog & mouth & house \\
\hline
圣 & 鼻 & 时 & 鬼 & 命 \\
god & nose & time & devil & life \\
\hline
光 & 终 & 案 & 血 & 边 \\
light & end & table & blood & side \\
\hline
背 & 儿 & 钱 & 舌 & 污 \\
back & son/child & money & tongue & dirt \\
\hline
\end{tabular}}
\caption{\label{tabletrans1} Translation for Chinese}
\end{table}

\begin{table}[htb]
\centerline{
\begin{tabular}{|c|c|c|c|c|}
\hline
Hand & Auge & Kopf & Ohr & Herz \\
hand & eye & head & ear & heart \\
\hline
Fu\ss & Tag & Finger & Gott & Weg \\
foot & day & finger & god & way \\
\hline
Wort & Bein & Arsch & Nase & Hals \\
word & leg & ass & nose & neck \\
\hline
Wasser & Zeit & Welt & Teufel & Leben \\
water & time & world & devil & life \\
\hline
Hund & Mund & Licht & Himmel & Mann \\
dog & mouth & light & sky & man \\
\hline
Tod & Wind & Boden & Luft & Haus \\
death & wind & earth/bottom & air & house \\
\hline
Leib & T\"{u}r & Ende & Tisch & Blut \\
body & door & end & table & blood \\
\hline
Gesicht & Seite & R\"{u}cken & Kind & Sack \\
face & side & back & child & bag/sack \\
\hline
Feuer & Seele & Geld & Zunge & Brot \\
fire & soul & money & tongue & bread \\
\hline
Maul & Tasche & Dreck & St\"{u}ck & Ei \\
mouth & pocket/bag & dirt & piece & egg \\
\hline
Sinn & Haar & Pferd &  & \\
meaning & hair & horse &  & \\
\hline
\end{tabular}}
\caption{\label{tabletrans2} Translation for German}
\end{table}

\begin{table}[htb]
\centerline{
\begin{tabular}{|c|c|c|c|c|}
\hline
mano & pie & ojo & ser & boca \\
hand & foot & eye & being & mouth \\
\hline
vida & cabeza & dios & d\'{i}a & cara \\
life & head & god & day & face \\
\hline
agua & palabra & cuenta & pelo & punto \\
water & word & bill & hair & point \\
\hline
mundo & culo & cosa & sangre & parte \\
world & ass & thing & blood & part \\
\hline
huevo & coraz\'{o}n & paso & hora & pan \\
egg & heart & step & hour & bread \\
\hline
alma & nariz & tiempo & puerta & perro \\
soul & nose & time & door & dog \\
\hline
palo & gato & pata & tierra & aire \\
stick & cat & foot & earth & air \\
\hline
cuerpo & punta & madre & brazo & gracia \\
body & point & mother & breast & mercy \\
\hline
dedo & diente & espalda & vez & a\~{n}o \\
doubt & tooth & back & time & year \\
\hline
lengua & medio & oreja & vuelta & vista \\
tongue & half/average & ear & rotation & view \\
\hline
\end{tabular}}
\caption{\label{tabletrans3} Translation for Spanish}
\end{table}

\end{document}
 
