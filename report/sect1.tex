\section{Introduction}
Chinese idioms (成语), are multi-word expressions in Chinese language, mostly derived from history or classical literature. There are many resources for Chinese idioms, with the collection size ranges from 5000 to 20000, depending on the definition.

% Most of the idioms consist of four characters/words, carrying more meaning than the sum of the words, since they usually refer to a story, one can not understand the meaning of the idiom without knowing the underlying story.

The origin of the idioms ranges starts from more than 2000 years ago, therefore the use of syntax and meaning of vocabulary can be quite different from modern standard Chinese. 

In this report\footnote{All the related files and codes used in the report can be downloaded at \url{https://github.com/EggplantElf/chinese_idioms}.}, we focus our research on two aspect of Chinese idioms. 
One is the frequent nouns in Chinese idioms. In concrete, we first annotated the part-of-speeches of the words used in the idioms, then sorted the nouns by their frequencies, and compared them to the most frequent nouns in German and Spanish \citep{mahlow:2013, Duden:2008, Seco:2004}. 
The other aspect is the frequent syntax patterns of Chinese idioms, since we have annotated the part-of-speeches, we could explore the structures and how they differ from modern Chinese syntax.

The results in the first part confirmed that most of the frequent nouns are shared in the MWEs of the three languages, we then tried to explain the presence and absence of some words from different aspects.
In the second part we found the most remarkable feature in the syntactic structure, namely the parallelism. 

The report is structured as follows: Section~\ref{anno} describes the annotation guidelines and procedures, Section~\ref{nouns} shows the statistics of the top nouns in Chinese idioms and the comparison to German and Spanish MWEs, Section~\ref{discuss} discusses the possible reasons for the difference between them, Section~\ref{pattern} analyzes the syntactic structures of the idioms. A glossary of the frequent words are given in the appendix.
