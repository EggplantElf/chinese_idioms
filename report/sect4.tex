\section{Discussion}\label{discuss}

In this section, we try to find the reasons for the differences in the distribution of frequent nouns in Chinese, German and Spanish. 

\subsection{Stylistic Reason}
It is significant that body parts are more frequent in German and Spanish but nature phenomena in Chinese. One reason may be that the MWEs in German and Spanish are taken from colloquial situations, where the body nouns are best to describe and make metaphor for daily life. In contrast, Chinese idioms derives mostly from literatures, where the nature are more preferred. 

More on this kind of preferences, one may find many nouns like 花(flower), 雨(rain), 云(cloud), 月(moon), 山(mountain) as frequent Chinese nouns. In ancient Chinese literature, such image words are usually used for making metaphor for emotions. But these words are not as the same usage in German and Spanish. 

\subsection{Religion Reason}
In German and Spanish, we all found word for god (German: Gott, Spanish: dios), but we didn't find any related noun in Chinese. The simple reason for this is the religion. God is a noun for Christian, while Ancient Chinese had religion of Buddhism. The noun 佛(Buddha) is also in the Chinese list but with a relatively low frequency (the 223th, with 25 occurrences). This low frequency is also reasonable in our Chinese idiom dictionary because the idioms came from 200B.C. to 20th century, but Buddhism widely spread in Ancient China after 500A.D., and it never became the official national religion in any dynasty of China. Another traditional religion in Ancient China was Taoism, the related noun for it was 道. Unfortunately, this noun is polysemous with other meanings as road, method or doctrine. Although 道 is very frequent in the corpus (15th noun with 127 occurrences), we cannot assign it as a religion related noun. 

\subsection{Natural Reason}
We also found many nouns for animals in all the three languages. But the kind of frequent nouns are different in distributions among the languages. We found dog in all languages (Chinese: 狗/犬, German:Hund, Spanish: perro), but we had tiger(虎) only in Chinese, and cat(gato) only in Spanish, and also the fictional creature, dragon(龙), only in Chinese. 
For the animal dog, it appears in all countries of the three, and also very common in the daily life. So idioms tend to contain them in the sayings. But for the animal tiger, which lives only in Asian, surely does not appear in WMEs in German and Spanish. 
% For the animal cat, it is also a common animal in the three countries, but only a frequent word in Spanish. One reason may be that cat is one of the most favorite animal in Spain. But for the culture reason, cat is usually used as evil in Chinese. So you won't find many Chinese idioms containing cats, but a lot in Spanish. 

We also found an interesting noun distribution for food. We have bread (German: Brot, Spanish: pan) in German and Spanish, but no matching noun in Chinese. It is known that bread is not a traditional food in China, but it is in Europe. So it is obvious that this noun is so frequent in German (45th with 37 occurrence) and Spanish (25th with 58 occurrence), but no occurrence in Chinese. Instead of bread, we have 饭(rice/food, 193th) and 米(rice, 299th) in the Chinese list, but with rather low frequency, again, because in literatures, food are less concerned than in colloquial expressions. 