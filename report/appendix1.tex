\section{Something}

\subsection{Results for Top Nouns in Chinese and German}\label{firstresult}
After annotation, we analyzed all idioms in the dictionary and found the most frequent nouns occurred. Then we aligned the top 50 Chinese nouns and top 50 German nouns to see if there is any difference in distribution. 

We found top 50 German nouns that can be aligned to the top 50 Chinese nouns and shown in the table\ref{tabletncg}. The table shows the order for each word in the top list (e.g. Mann is the 24th in German list). From the table, we can find 29 out of 50 nouns are aligned. The number is reasonable since we have only 50 vs. 50 alignment. In the table, it is shown that the nature phenomena are more frequent while they are less frequent in German. 

\begin{table}[htb]
\caption{\label{tabletncg} {\it Top 50 Nouns Alignment for Chinese and German (ordered in Chinese)}}
\centerline{
\begin{tabular}{|c|c|c|c|c|c|c|c|c|c|}
\hline
\hline
1 & 2 & 3 & 4 & 5 & 6 & 7 & 8 & 9 & 10 \\
人 & 心 & 天 & 风 & 山 & 头 & 地 & 水 & 日 & 马 \\
\hline
24 & 5 & 23 & 25 &  & 3 & 26 & 16 & 7 & 50 \\
Mann & Herz & Himmel & Wind &  & Kopf & Boden & Wasser & Tag & Pferd \\
\hline
\hline
11 & 12 & 13 & 14 & 15 & 16 & 17 & 18 & 19 & 20 \\
金 & 目 & 事 & 口 & 道 & 手 & 云 & 月 & 气 & 门 \\
\hline
36 & 2 &  & 21 & 10 & 1 &  &  & 27 & 42 \\
Geld & Auge &  & Mund/Maul & Weg & Hand &  &  & Luft & T\"{u}r \\
\hline
\hline
21 & 22 & 22 & 23 & 24 & 24 & 24 & 25 & 26 & 26 \\
足 & 虎 & 面 & 火 & 身 & 流 & 子 & 玉 & 声 & 情 \\
\hline
6 &  & 31 & 34 & 29 &  & 33 &  &  & \\
Fu\ss &  & Gesicht & Feuer & Leib &  & Kind &  &  & \\
\hline
\hline
27 & 28 & 28 & 28 & 29 & 30 & 31 & 32 & 32 & 33 \\
食 & 花 & 发 & 名 & 海 & 雨 & 世 & 家 & 年 & 龙 \\
\hline
  &  & 45 &  &  &  & 18 & 28 &  & \\
  &  & Haar &  &  &  & Welt & Haus &  & \\
\hline
\hline
34 & 35 & 36 & 37 & 38 & 39 & 40 & 41 & 41 & 42 \\
力 & 意 & 色 & 衣 & 神 & 眼 & 耳 & 国 & 文 & 公 \\
\hline
  & 40 &  &  & 35 & 2 & 4 &  & 11 & \\
  & Sinn &  &  & Seele & Auge & Ohr &  & Wort & \\
\hline
\hline
\end{tabular}}
\end{table}


Another table aligning Chinese nouns to top 50 German nouns is shown in \ref{tabletngc}. Since that we use a longer list for Chinese (containing top 500 Chinese nouns), we have 41 out of 50 matched. Due to this huge list and translation, we have more than one Chinese noun matched to a German noun, but we also provide the order for each words. It is shown that the body parts comes more than other nouns in German, while most of them can also be found in Chinese with relatively high frequency. 

\begin{table}[htb]
\caption{\label{tabletngc} {\it Top 50 Nouns Alignment for Chinese and German (ordered in German)}}
\centerline{
\begin{tabular}{|c|c|c|c|c|c|c|c|c|c|}
\hline
\hline
1 & 2 & 3 & 4 & 5 & 6 & 7 & 8 & 9 & 10 \\
Hand & Auge & Kopf & Ohr & Herz & Fu\ss & Tag & Finger & Gott & Weg \\
\hline
16 & 12/46 & 5 & 47 & 2 & 21/132 & 9 &  & 437 & 15/101 \\
手 & 目/眼 & 头 & 耳 & 心 & 足/脚 & 日 &  & 圣 & 道/路 \\
\hline
\hline
11 & 12 & 13 & 14 & 15 & 16 & 17 & 18 & 19 & 20 \\
Wort & Bein & Arsch & Nase & Hals & Wasser & Zeit & Welt & Teufel & Leben \\
\hline
48/77/166/229 &  &  & 464 &  & 8 & 52 & 37 & 118 & 82 \\
文/语/字/词 &  &  & 鼻 &  & 水 & 时 & 世 & 鬼 & 命 \\
\hline
\hline
20 & 21 & 22 & 23 & 24 & 25 & 25 & 26 & 27 & 28 \\
Hund & Mund & Licht & Himmel & Mann & Tod & Wind & Boden & Luft & Haus \\
\hline
105/151 & 14/176 & 67 & 3 & 1 &  & 4 & 7 & 19 & 38/229 \\
狗/犬 & 口/嘴 & 光 & 天 & 人 &  & 风 & 地 & 气 & 家/屋 \\
\hline
\hline
29 & 29 & 30 & 31 & 31 & 31 & 32 & 33 & 33 & 34 \\
Leib & T\"{u}r & Ende & Tisch & Blut & Gesicht & Seite & R\"{u}cken & Kind & Sack \\
\hline
25 & 20 & 166 & 360 & 128 & 22 & 312 & 138 & 25/218 & \\
身 & 门 & 终 & 案 & 血 & 面 & 边 & 背 & 子/儿 & \\
\hline
\hline
34 & 35 & 36 & 36 & 37 & 37 & 38 & 38 & 39 & 40 \\
Feuer & Seele & Geld & Zunge & Brot & Maul & Tasche & Dreck & St\"{u}ck & Sinn \\
\hline
24 & 45 & 11/332 & 89 &  & 14/176 &  & 464 &  & 42 \\
火 & 神 & 金/钱 & 舌 &  & 口/嘴 &  & 污 &  & 意 \\
\hline
\hline
\end{tabular}}
\end{table}


\subsection{Results for more language}
After getting the list for frequent nouns in Spanish MWE, we also compared them with our Chinese list. The results are shown in figures so that it is easier to see the difference for distribution. 

%\figure{figure1}\label{fig1}

The figure\ref{fig1} shows the frequency for the top 20 Chinese nouns aligned with German nouns (up to 25th Chinese noun). The y-axis shows translation-Chinese-German for each item. As is discussed in \ref{firstresult}, body parts are more frequent in German compared to nature phenomena in Chinese. 

%\figure{figure2}\label{fig2}

The figure\ref{fig2} shows the figure for Chinese and Spanish. The alignment is harder than with German since we used up to 47th Chinese noun to get 20 matches. Compared to German version, although less match in most frequent nouns, but the distribution for the matches are more similar in respect of the absolute frequency number. 

%\figure{figure3}\label{fig3}

%\figure{figure4}\label{fig4}

Figure\ref{fig3} and \ref{fig4} are the alignment for German with Chinese and Spanish with Chinese. The result shows no significant difference. But this time, German was matched up to 39th while Spanish was up to 48th. Compared to \ref{fig1}, the alignment is harder here. One interesting thing is that the frequency in Chinese are much higher than the other two languages for the same noun. 