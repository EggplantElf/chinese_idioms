\section{Pattern Analysis}\label{pattern}
After analyzing the frequent words in different languages, we also tried to find frequent syntactic pattern in Chinese idioms. One important feature of Chinese idioms is that most of the idioms are constructed with four characters/words, it is interesting to look into the structures.

\subsection{Pattern Extraction and Results}
After annotating the part-of-speeches for frequent words (frequency larger than 9), we look only for those idioms containing the frequent words in the dictionary. The results are shown in Table~\ref{tablepaex}. In the `Pattern' column of the table, n for noun, v for verb, a for adverb and adjective, and i for numeral. 

\begin{table}[htb]
\centerline{
\begin{tabular}{|c|c|c|c|}
\hline
Order & Pattern & Frequency & Examples \\
\hline
1 & vnvn & 480 & 争权夺利	, 借刀杀人 \\
\hline
2 & nnnn & 236 & 人山人海, 唇枪舌剑 \\
\hline
3 & anan & 189 & 大头小尾	, 白山黑水 \\
\hline
4 & nvnv & 175 & 云消雾散	, 土崩瓦解 \\
\hline
5 & nana & 143 & 人多嘴杂	, 地久天长 \\
\hline
6 & nnvn & 121 & 井井有条, 日月入怀 \\
\hline
7 & anvn & 118 & 先声夺人	, 小鸟依人 \\
\hline
8 & annn & 94 & 满城风雨, 满面春风 \\
\hline
9 & vnvv & 87 & 临阵脱逃, 倾巢出动 \\
\hline
10 & vnnn & 75 & 看人眉眼	, 过眼烟云 \\
\hline
11 & anav & 70 & 上情下达	, 冷眼静看 \\
\hline
11 & inin & 70 & 一丝一毫	, 千山万水 \\
\hline
13 & nnan & 69 & 道边苦李	, 鼎鼎大名 \\
\hline
14 & vvvv & 65 & 生死存亡	, 起承转合 \\
\hline
14 & vava & 65 & 居高临下	, 弃暗投明 \\
\hline
16 & nvnn & 62 & 天随人愿	, 声振林木 \\
\hline
17 & vnan & 61 & 垂名青史	, 识途老马 \\
\hline
18 & avav & 56 & 南征北战, 寡闻少见 \\
\hline
19 & nnav & 55 & 人神共愤	, 原形毕露 \\
\hline
20 & nnvv & 49 & 利害得失	, 儿女成行 \\
\hline
20 & vvvn & 49 & 招摇过市, 凿凿有据 \\
\hline
\end{tabular}}
\caption{\label{tablepaex} Top 20 Frequent Patterns and Example Idioms (ordered in frequency)}
\end{table}

\subsection{Discussion for Patterns}
Table~\ref{tablepaex} shows top 20 frequent pattern in the dictionary. It is obvious that most of the high frequency patterns are parallelism, i.e. in the form of `AABB' or `ABAB', and the two parallel parts are both syntactically and semantically similar.

Parallelism is a very salient feature in the ancient Chinese literature, therefore also in the derived idioms. Since the words in ancient Chinese are monosyllabic, there could be many ambiguities. With parallelism, it is much easier to recognize the intended meaning.
This also explains why the frequencies of the nouns in Chinese idioms are generally higher than German and Spainish MWEs though the size of dictionary are similar, simply because there are usually more than one nouns in each Chinese idioms.

% When finding example for patterns and checking the results, we found that it was hard to find an example for some frequent pattern, for example, `annn' and `vvvn'. One reason for this is that most idioms are in parallelism, for those unbalanced patterns, it shouldn't be frequent in the idioms. Another reason is the limitation of NAIV annotation. We only annotated frequent words and didn't provide different annotations for polysemy. So there are many idioms belong to wrong pattern because of the annotation mistake. 

We also tried to use wild cards to extract more kind of patterns, but due to the four-word structural limitation, the results are not satisfying. In the future, we could produce more analysis after getting a more fine-grained annotated corpus for Chinese idioms. 